\section*{Project Summary}

\paragraph*{Overview}
Property-based testing (PBT) is an advanced software testing
methodology where users write executable specifications of their code
and an automated test harness checks that they hold for many randomly
generated test inputs.  Its bug-finding power arises from its ability
to exercise a wide range of system behaviors---both expected and
unexpected---with minimal user guidance.
%
From its roots in the QuickCheck library in Haskell, PBT has grown
into a standard tool across the functional programming community, and it
has begun to make inroads into industrial practice
% and it is more recently being incorporated into testing processes
in companies such as Quviq, Galois, and Amazon.\iflater\todo{Who else??}\fi{}
% ... where it finds lots of bugs!
%
The goal of this project is to accelerate this transition
by identifying and addressing key challenges in
present-day PBT tools.

What are these challenges?
A need-finding study of PBT at Jane
Street, a Wall Street firm with a focus on software
technology as a competitive advantage, found developers enthusiastic
about its {\em usefulness} but
sometimes frustrated by issues with {\em usability}.
%
The study identified usability issues in both of the major components
of PBT tooling---{\em specification} of properties and {\em
  generation} of random inputs. Furthermore, it identified challenges
around another aspect of PBT that has received less attention in the
research literature: how programmers can {\em validate} the
effectiveness of testing.

% At the level of {\em specification}, developers wished for better
% support for expressing common requirements such as ``this
% high-performance implementation of a service must behave exactly the
% same as that reference model'' and for articulating properties of
% poorly modularized code.
% %
% Concerning the other main component of PBT, the {\em generation} of
% random test cases, \todo{...}
% %
% Developers also need better {\em validation} tools for checking
% that their testing is actually effective.\todo{blah}
% %
% Finally, better {\em education} of potential PBT users is critically
% required.

% \amh{How about the following framing? In this project, we define a practice of
% usable property-based testing, and conduct foundational research in PL and HCI
% to bring about this practice. We define usable property based-testing as having
% two main components. First, providing visibility into what can be extremely
% complex test outcomes. Second, supporting the efficient expression of
% tests.\ldots{} And then we can jump into talking about the challenges that we
% plan to solve in specification, generation, validation, and education.}.
% %
% At the level of {\em specification}, developers wished for better
% support for expressing common requirements such as ``this
% high-performance implementation of a service must behave exactly the
% same as that reference model'' and for articulating properties of
% poorly modularized code.
% %
% Concerning the other main component of PBT, the {\em generation} of
% random test cases, \todo{...}
% %
% Developers also need better {\em validation} tools for checking
% that their testing is actually effective.\todo{blah}
% %
% Finally, better {\em education} of potential PBT users is critically
% required.

%  and ``shrinking'' of failing
% tests to minimal counterexamples,

To make progress, two mostly disjoint research areas
must be brought to bear.  PBT itself is grounded in domain-specific
languages and formal methods---traditional PL topics.  Usability, on
the other hand, is the domain of human-computer interaction.
%
This four-year project will combine insights, methods, and tools from
PL and HCI to advance the state of the art in all three components of
PBT---specification, generation, and validation---to make PBT
significantly more usable across a broad community of software developers.

\smallskip

\noindent{\bf Keywords:} Property-based testing, human-computer
interaction, programming languages, usability.

\paragraph*{Intellectual Merit}
The project will advance knowledge along five interconnected axes.
First, it will establish a firm {\em foundation} for HCI-informed
research on PBT, supplementing past and ongoing user studies with
broader surveys of PBT across the software industry and real-time
observations of developers using PBT tools.  Second, it will \todo{...}


 in specification, generation, and validation

% The statement on intellectual merit should describe the potential of the proposed activity to advance knowledge.

 (1) deploy tools from HCI to better
understand the challenges to wider adoption of PBT, (2) combine PL and
HCI methods to build solutions, and (3) develop educational materials
to make PBT concepts and tools accessible to a broad community of
university students and industrial developers.

\begin{itemize}
\item Establish a picture of the challenges
faced by developers when using PBT informed by rigorous user-centered research.
\item Develop a new abstraction for random data generators, {\em
  reflective generators}, enabling better (manual or automatic) tuning of test case
distributions, ...
\item benchmark suite
\item Develop a conceptual framework and concrete tools for validating
the effectiveness of testing (specifics...),
visualizing and tuning generator distributions,
interactive tools for shrinking,
converting counterexamples to readable regression tests
\item Evaluate our results through further user studies...
\end{itemize}

\paragraph*{Broader Impacts}
\begin{itemize}
\item make software better, more robust, more reliable, more secure(??)
\item (describe educational projects)
\item Develop tools that significantly reduce barriers to entry
\item Work with undergrads, etc.
\end{itemize}

% The statement on broader impacts should describe the potential of the proposed activity to benefit society and contribute to the achievement of specific, desired societal outcomes.
