\section*{Project Summary}

\todo{Recheck in GPG}

\paragraph*{Overview}
Property-based testing (PBT) is an advanced software testing
methodology where users write executable specifications of their code
and an automated test harness checks that they hold for many randomly
generated test inputs.  From its roots in the QuickCheck library in
Haskell, PBT has grown into a standard tool across the functional
programming community, and it has begun to make inroads into
industrial practice in companies such as Quviq, Galois, Amazon, and
Jane Street.
%
However, despite its well-established bug-finding power, uptake of PBT
in industry has been slow.  A need-finding study of PBT users at Jane
Street found them enthusiastic about its {\em usefulness} but
sometimes frustrated by issues with {\em usability}.
%
\amh{How about the following framing? In this project, we define a practice of 
usable property-based testing, and conduct foundational research in PL and HCI 
to bring about this practice. We define usable property based-testing as having 
two main components. First, providing visibility into what can be extremely 
complex test outcomes. Second, supporting the efficient expression of 
tests.\ldots{} And then we can jump into talking about the challenges that we 
plan to solve in specification, generation, validation, and education.}.
%
At the level of {\em specification}, developers wished for better
support for expressing common requirements such as ``this
high-performance implementation of a service must behave exactly the
same as that reference model'' and for articulating properties of
poorly modularized code.
% 
Concerning the other main component of PBT, the {\em generation} of
random test cases, \todo{...}
%
Developers also need better {\em validation} tools for checking
that their testing is actually effective.\todo{blah}
%
Finally, better {\em education} of potential PBT users is critically
required.

%  and ``shrinking'' of failing
% tests to minimal counterexamples,  

\begin{itemize}
\item PBT originates in PL; HCI is all about usability.  We need
insights from both to close the gap.
\item The four-year program of research and education in
PBT proposed here will combine PL and HCI methods to make PBT usable by a broad
community of industrial developers and students.
\end{itemize}

Keywords: Property-based testing, human-computer interaction,
usability.

\paragraph*{Intellectual Merit}
% The statement on intellectual merit should describe the potential of the proposed activity to advance knowledge.
\begin{itemize}
\item Establish a picture of the challenges
faced by developers when using PBT informed by rigorous user-centered research.
\item Develop a new abstraction for random data generators, {\em
  reflective generators}, enabling better (manual or automatic) tuning of test case
distributions, ...
\item benchmark suite
\item Develop a conceptual framework and concrete tools for validating
the effectiveness of testing (specifics...),
visualizing and tuning generator distributions,
interactive tools for shrinking,
converting counterexamples to readable regression tests
\item Evaluate our results through further user studies...
\end{itemize}

\paragraph*{Broader Impacts}
\begin{itemize}
\item make software better, more robust, more reliable, more secure(??)
\item (describe educational projects)
\item Develop tools that significantly reduce barriers to entry
\item Work with undergrads, etc.
\end{itemize}

% The statement on broader impacts should describe the potential of the proposed activity to benefit society and contribute to the achievement of specific, desired societal outcomes.
