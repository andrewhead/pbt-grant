\section*{Project Summary}

\todo{Recheck in GPG}

\paragraph*{Overview}
\begin{itemize}
\item PBT is promising.  Starting to move from academia into industry
at places like Quviq, Galois, Amazon, and JS.
\item A study was carried out at JS to understand how to accelerate the
process.
\item Study found that people find PBT very {\em useful}, but there are
some challenges around {\em usability}.  List three categories, also
education.
\item PBT originates in PL; HCI is all about usability.  We need
insights from both to close the gap.
\item The four-year program of research and education in
PBT proposed here will combine PL and HCI methods to make PBT usable by a broad
community of industrial developers and students.
\end{itemize}

Keywords: Property-based random testing, human-computer interaction,
usability.

\paragraph*{Intellectual Merit}
% The statement on intellectual merit should describe the potential of the proposed activity to advance knowledge.
\begin{itemize}
\item Establish a rigorous, data-driven picture of the challenges
faced by developers when using PBT.
\item Develop a new abstraction for random data generators, {\em
  reflective generators}, enabling better (manual or automatic) tuning of test case
distributions, ...
\item benchmark suite
\item Develop a conceptual framework and concrete tools for validating
the effectiveness of testing (specifics...),
visualizing and tuning generator distributions,
interactive tools for shrinking,
converting counterexamples to readable regression tests
\item Evaluate our results through further user studies...
\end{itemize}

\paragraph*{Broader Impacts}
\begin{itemize}
\item (describe educational projects)
\item Develop tools that significantly reduce barriers to entry
\item Work with undergrads, etc.
\end{itemize}

% The statement on broader impacts should describe the potential of the proposed activity to benefit society and contribute to the achievement of specific, desired societal outcomes.
