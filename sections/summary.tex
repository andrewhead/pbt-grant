\section*{Project Summary}

% \newcommand{\summarysection}[1]{\subsubsection*{#1}}

\newcommand{\summarysection}[1]{\paragraph*{#1.}}

\summarysection{Overview}
Property-based testing (PBT) is an advanced software engineering
methodology where users write executable formal specifications of system
components
and an automated test harness checks these specifications against
many automatically generated inputs.  The bug-finding power of PBT
arises its ability both to work with rich specifications and
to exercise a wide range of system behaviors---both
expected and unexpected---with minimal user guidance.
%
From its roots in the Haskell QuickCheck library, PBT has become
the testing method of choice across much of the functional programming
community; it has also begun to make inroads into industrial practice
at companies such as Quviq, Galois, Stripe, and
Amazon.\footnote{\forreaders{Any other big companies?  Notable open-source
  projects?}\iflater\bcp{Remove me!}\fi}}
%
The goal of this proposal is to accelerate this transition
by addressing key challenges in
present-day PBT tools.

What are these challenges?
A need-finding study of PBT at Jane
Street, a Wall Street firm with a focus on software
technology as a competitive advantage, found developers enthusiastic
about its {\em usefulness} but
sometimes frustrated with its {\em usability}.
%
Indeed, the study identified usability as an issue in both major aspects
of PBT methods---{\em specification} of properties and {\em
  generation} of random inputs. Moreover, it revealed challenges
around another aspect of PBT that has so far received less attention from
researchers: how programmers can {\em validate} the
effectiveness of testing.

% At the level of {\em specification}, developers wished for better
% support for expressing common requirements such as ``this
% high-performance implementation of a service must behave exactly the
% same as that reference model'' and for articulating properties of
% poorly modularized code.
% %
% Concerning the other main component of PBT, the {\em generation} of
% random test cases, \todo{...}
% %
% Developers also need better {\em validation} tools for checking
% that their testing is actually effective.\todo{blah}
% %
% Finally, better {\em education} of potential PBT users is critically
% required.

% \amh{How about the following framing? In this project, we define a practice of
% usable property-based testing, and conduct foundational research in PL and HCI
% to bring about this practice. We define usable property based-testing as having
% two main components. First, providing visibility into what can be extremely
% complex test outcomes. Second, supporting the efficient expression of
% tests.\ldots{} And then we can jump into talking about the challenges that we
% plan to solve in specification, generation, validation, and education.}.
% %
% At the level of {\em specification}, developers wished for better
% support for expressing common requirements such as ``this
% high-performance implementation of a service must behave exactly the
% same as that reference model'' and for articulating properties of
% poorly modularized code.
% %
% Concerning the other main component of PBT, the {\em generation} of
% random test cases, \todo{...}
% %
% Developers also need better {\em validation} tools for checking
% that their testing is actually effective.\todo{blah}
% %
% Finally, better {\em education} of potential PBT users is critically
% required.

%  and ``shrinking'' of failing
% tests to minimal counterexamples,

To make PBT more usable, two largely disjoint
research areas must be
brought to bear.  PBT itself is grounded in domain-specific languages
and formal methods---traditional topics in the field of Programming Languages.
Usability, on the other hand, is the domain of Human-Computer Interaction.
%
This five-year project will combine insights, methods, and tools from
PL and HCI to advance the usability of PBT along all three of its
axes---specification, generation, and validation\iflater\todo{We may
  need to reorder these, depending on what happens with the global
  structure.}\fi---and bring it to a
broad community of software developers.

\smallskip

\noindent{\bf Keywords:} Programming languages; human-computer
interaction; property-based testing; usability.

\summarysection{Intellectual Merit}
The project will advance knowledge along four interconnected axes.
%
First, it will establish a firm {\bf\em foundation} for HCI-informed
research on PBT, supplementing our past and ongoing user studies with
broader surveys of PBT across the software industry and real-time
observations of developers interacting with PBT.
%
Second, it will offer developers more usable {\bf\em specification} tools,
including a language for stating temporal properties over internal
program states and automation for model-based testing of
modular abstractions.
%
Third, it will explore a novel abstraction for random input {\bf\em
  generation} that enables a range of use cases---generating inputs
satisfying validity conditions, mutating inputs to explore the space
of similar inputs, and manually or automatically tuning a random
generator's distribution based on examples or code coverage.
%
And fourth, it will develop new tools for effective {\bf\em
  interaction} between developers and their tests, includingools for
evaluating the effectiveness of testing, for helping programmers pin
down the causes of failures, and for visualizing generated data
distributions to support comprehension and tuning.

% The project will advance knowledge along four interconnected axes.
% %
% First, it will establish a firm {\bf\em foundation} for HCI-informed
% research on PBT, supplementing past and ongoing user studies with
% broader surveys of PBT across the software industry and real-time
% observations of developers interacting with PBT.
% %
% Second, it will offer developers more usable {\bf\em specification} tools,
% including a language for stating temporal properties over internal
% program states and automation for model-based testing of
% modular abstractions.
% %
% Third, it will explore a novel abstraction for random input {\bf\em
%   generation} that enables a range of use cases---generating inputs
% satisfying validity conditions, mutating inputs to explore the space
% of similar inputs, and manually or automatically tuning a random
% generator's distribution based on examples or code coverage.
% %
% And fourth, it will develop new interactive tools for rapid {\bf\em
%   validation} of the effectiveness of testing, for helping programmers pin
% down the causes of failures, and for visualizing generated data
% distributions to support comprehension and tuning.

%  (1) deploy tools from HCI to better
% understand the challenges to wider adoption of PBT, (2) combine PL and
% HCI methods to build solutions, and (3) develop educational materials
% to make PBT concepts and tools accessible to a broad community of
% university students and industrial developers.

\summarysection{Broader Impacts}
Two further threads of activity, coordinated with the rest and integral to the
project's aims, will
%
be to support the {\bf\em diffusion} of PBT tools and
methodologies from academia into industry through targeted engineering efforts
%
and to advance PBT {\bf\em education} in
both universities and industry.  We will support and strengthen existing
open-source PBT tools in popular programming languages and enrich their
capabilities with re-engineered research products from the four themes described
above;
%
and we will
develop materials for teaching industry programmers how to identify
high-leverage situations for using PBT, curricula for teaching mature
and powerful PBT practices to undergraduate students, and tooling to help
PBT beginners write their first properties.
%
A general-audience article on PBT for CACM will also help broaden the
reach of PBT.\iflater\bcp{Maybe that last is too detailed for the
  summary?}\fi

Work on the project will include and elevate undergraduate
researchers, including some from diverse backgrounds that will be
funded through a new NSF-REU. These undergraduates will work
closely with the two PIs and four Ph.D.{} students in an interconnected and supportive
research environment.

The ultimate goal, through education, publications, and open-source
tools, is to make property-based testing a standard tool on every
software developer's testing toolbelt.  Better testing, in turn, will
lead to software systems of every description that are less expensive, more
robust, and more reliable.


% The statement on broader impacts should describe the potential of the proposed activity to benefit society and contribute to the achievement of specific, desired societal outcomes.
