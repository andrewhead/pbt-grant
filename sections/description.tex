\section{Project Description}
\subsection{Introduction}

\section{Background and Related Work}

\section{PIs' Prior Qualifications}

\section{Technical Foundations}

This is a placeholder for the ``theory'' or ``known unknowns'' section.

\section{Need-Finding}

\textbf{Pardon the provisional content, this section is a work in progress.}

Part 1 of the ``unknown unknowns''.
Jane Street study, preliminary study, observations, surveys.

In this aim, we plan to set course for research in PBT, both broadly and for the 
future directions of our group. We will conduct a series of formative studies, 
each addressing one of three main goals:

\subsection{Understanding Challenges to Using PBT Tools}

The first step is to conduct qualitative interviews to understand the challenges 
that programmers face when using PBT tools.

Here I will elaborate on the design of the Jane Street study.

\subsection{Characterizing the Potential Reach for PBT Tools}

We will conduct a survey with the purpose of understanding the impact that 
improved PBT tools could have in industry, if major usability issues were 
addressed. We see this survey as crucial in understanding the amount of 
resources that should be devoted to PBT research. Furthermore, this 
questionnaire will help shed light into which of the usability challenges from 
the interview study, if addressed, are most likely to impact a broad set of 
current and prospective users of PBT tools.

\subsection{Understanding the Structure of PBT Tasks in Detail}

The next step is to more deeply characterize the obstacles faced with specific 
tools and tasks through close observation. We anticipate conducting observations 
of participants formulating properties and creating generators, as we expect 
that close observation of developers performing these tasks will yield yet 
additional detail about ways that developers are supported and not supported by 
the tools they use today to a level of depth we will not achieve with the 
interviews.

\section{Tool Design and Development}

\textbf{Pardon the provisional content, this section is a work in progress.}

Part 2 of the ``unknown unknowns''.
Speculative work.

Upon the advanced technical foundation from Aim 1 and the refined understanding 
of programmers' needs from Aim 2, our final aim (Aim 3) will focus on the 
development of programmer-facing tools that allow them to leverage properties in 
testing their code more efficiently and effectively.

While the specific focus of our tool design and research efforts will be 
continually refined on the basis of what we learn from Aim 2, below we detail 
several directions that we expect to lead to the design of tools that are both 
innovative within the research community, as well as potentially impactful, 
drawing on the lessons learned from the preliminary need-finding research we 
have done to date as well as our own intuitions, and as critical users and 
engaged members of the communities for these tools.

List out the directions we will pursue in tool design and development\ldots{}

\textbf{To discuss:}

\begin{itemize}

\item Some of the tools we propose in this section may depend on a technical 
background we do not have on this team. For instance, property generation.  
Should we be keeping the scope of the proposed work to those where we as a team 
have expertise in the backend technologies?

\item We should ask Jane Street for a letter of support indicating their 
willingness to collaborate on the need-finding activities.

\end{itemize}



\section{Education Plan}

\subsection{Broader Impacts of the Proposed Work}
The Project Description must contain, as a separate section within the narrative, a section labeled ``Broader
Impacts of the Proposed Work". This section should provide a discussion of the broader impacts of the proposed
activities. Broader impacts may be accomplished through the research itself, through the activities that are
directly related to specific research projects, or through activities that are supported by, but are complementary to 
the project. NSF values the advancement of scientific knowledge and activities that contribute to the
achievement of societally relevant outcomes. Such outcomes include, but are not limited to: full
participation of women, persons with disabilities, and underrepresented minorities in science, technology, engineering, and
mathematics (STEM); improved STEM education and educator development at any level; increased public
scientific literacy and public engagement with science and technology; improved well-being of individuals in
society; development of a diverse,globally competitive STEM workforce; increased partnerships between
academia, industry, and others; improved national security; increased economic competitiveness of the United
States; and enhanced infrastructure for research and education.

\subsection{Results from Prior NSF Support}
If any PI or co-PI identified on the project has received NSF funding (including any current
funding) in the past five years, in formation on the award(s) is required,
irrespective of whether the support was directly related to the proposal or not.
In cases where the PI or co-PI has received more than one award (excluding amendments),
they need only report on the one award most closely related to the proposal. Funding includes not just salary
support, but any funding awarded by NSF. The following information must be provided:\\

\noindent
\emph{\underline{Name of PI}}: NSF-Program (Award Number) ``Title of the Project'' (\$AMOUNT, PERIOD OF SUPPORT). 
{\bf Publications:} List of publications resulting from the NSF award. A complete bibliographic citation for each
publication must be provided either in this section or in the References Cited section of the proposal); if
none, state: ``No publications were produced under this award.'' {\bf Research Products:} evidence of research products 
and their availability, including, but not limited to: data, publications, samples, physical collections, software, 
and models, as described in any Data Management Plan.

% \subsection{Proposed Study}
% The Project Description should provide a clear statement of the work to be undertaken and must include:
% objectives for the period of the proposed work and expected significance; relation to longer-term goals of the PI's
% project; and relation to the present state of knowledge in the field, to work in progress by the PI under other
% support and to work in progress elsewhere.
% 
% The Project Description should outline the general plan of work, including the broad design of activities to be
% undertaken, and, where appropriate, provide a clear description of experimental methods and procedures.
% Proposers should address what they want to do, why they want to do it, how they plan to do it, how they will
% know if they succeed, and what benefits could accrue if the project is successful. The project activities may be
% based on previously established and/or innovative methods and approaches, but in either case must be well
% justified. These issues apply to both the technical aspects of the proposal and the way in which the project may
% make broader contributions.

