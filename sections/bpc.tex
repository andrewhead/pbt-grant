% \usepackage{tabularx}

\section*{Broadening Participation in Computing (BPC) Plan: Standalone}

The Computer and Information Science Department at the University of
Pennsylvania is in the final stages of developing a department-wide
BPC plan, which will be submitted to BPCnet for verification in the
new year.  The following section describes specific planned
BPC-related activities of PIs Head and Pierce.  Section 2 below
summarizes the current draft departmental plan, which the current PIs
are helping shape.

\subsection*{1. Investigator-Specific BPC Activities}

% \amh{Here is my attempt to split out the above 4 goals into 5 goals for which
% I think we can state measurable outcomes and a concrete strategy. I imagine we
% will massage this text into prose, rather than these structured lists, though
% I am using the structure for drafting the content.}
% \bcp{I think we are better off leaving the Departmental plan alone and
% laying out our personal plans in the second section.  (In part because
% people may improve the departmental one over the weekend!)}

% \textbf{Some proposed overarching goals?}: To work towards gender parity in our
% groups and the department more broadly, and representation of underrepresented
% minorities roughly reflective of Pennsylvania demographics (for us, perhaps
% focusing on the Black and African American community, which constitutes 11\% of
% the state population and 41\% of Philadelphia city population, but 5\% of our
% departmental student body, iirc).

% \begin{itemize}
% \item female PhD students -- goal could be to admit one(?) more each to
% our groups over the course of the grant
% \item 1200 curriculum -- efforts to make the course friendly to all --
% explicit diversity goals for TA hiring (51 TAs! including 26 women and
% 5 URMs out of 51 total TAs).  A good goal would be to consistently
% match or exceed the overall UPenn stats
% (\url{https://diversity.upenn.edu/diversity-at-penn/facts-and-figures}).
% For gender balance we are good (Penn population is 53\% female), but
% black+latinx is about 18\% of the Penn population and the TA
% proportion is just under 10\%, so we've got a ways
% to go there.
% \end{itemize}

\setcounter{footnote}{0}

\paragraph*{REU Student Mentoring}

\begin{enumerate}
\item {\bf Goal and context:} REPL (``Research Experiences for
undergraduates in Programming Languages'') is a new\footnote{More
  precisely: This REU has not been formally funded yet by the NSF, but
  we have recently been told, informally, by the cognizant program
  manager that he intends to proceed.  If things should fall through
  for any reason, we will submit an updated proposal again this year.}
Penn-hosted REU site that will catalyze the next generation of
programming languages research by preparing students for PhD programs
in programming languages and by increasing the number of
underrepresented persons in our field. To do this, REPL integrates
undergraduates with ``PLClub'', Penn’s research group in programming
languages.

\item {\bf Intended populations:} The undergraduates we select will
come from institutions across the US, and we will emphasize hiring
students whose backgrounds are underrepresented in programming
languages research. We will focus on expanding representation of
female, Black, and Latinx students, who are particularly
underrepresented in our
groups relative to the Pennsylvania population as a whole (the US Census Bureau
estimates PA is 51\% female, 12\% Black, and 8\% Hispanic or Latino).  The
program will support eight students  each
summer, for three consecutive summers.
\item {\bf Strategy:} Over the course of 10 weeks, the students will do
research, master prerequisite knowledge, read and analyze papers, understand
ongoing research trends, and learn how to prepare competitive PhD applications.
\item {\bf Measurement:}
REPL project leaders will work with the Computing Research Association’s Center for Evaluating the Research Pipeline (CERP), which is an evaluation center that has been contracted by NSF CISE to provide evaluation for REU Sites and REU Supplements. CERP’s evaluation work will focus on measuring the impact of REPL on students’ self-perceptions (e.g., self-efficacy; scientific identity), academic development (e.g., research productivity; skills proficiency) and professional aspirations (e.g., intentions to pursue graduate school; career goals). At the end of data collection, CERP will provide a report that summarizes evaluation results alongside a comparison group of responses collected from similar REUs. Demographic data and other student characteristics will be provided in the report with an intersectional lens when possible, enabling the project team to understand the impact of the project on different types of students. A designated liaison will be responsible for distributing the surveys and communicating with the CERP team, providing any information related to this project that is necessary for data collection and reporting.
\item {\bf PI Engagement:} Both PIs plan to work with REPL students.
\todo{Say which projects are particularly appropriate.  Mention that
  BCP wrote the text they will be studying from.}
\end{enumerate}

\paragraph*{TA Demographics for the Introductory CS Course}

\begin{enumerate}
\item {\bf Goal and context:} The intro computer science course for CS
majors at Penn, CIS 1200, is taken by about 750 students each year
from across the university.  Mixing functional, imperative, and
object-oriented programming in OCaml and Java, the course is heavily
assignment based, which requires a large staff of undergraduate TAs to
lead recitations and respond to questions on the class discussion
board.

\item {\bf Intended populations:} The 1200 instructors have for years
set a goal of hiring a team of TAs that at least matches the diversity of the
students taking the course. In recent years, we have had success expanding
representation of women, Black, and Latinx students (see ``Measurement''); we
will continue to do so. Our hope is that underrepresented students are
influential and visible members of the teaching staff, to increase the
likelihood that students see themselves in the teaching staff and that we
represent students equitably in course decision-making.  \amh{@Benjamin check
this}

\item {\bf Strategy:} TAs are hired at the end of each semester, for
the following semester.  We cast a wide net to get as many
applications as possible---inviting people in the current semester to
apply if they think they might enjoy it (even if they don't think of
themselves as ``typical TAs''), asking every current TA to encourage
individuals (especially diverse individuals) to apply, posting
repeated announcements, etc.  A sizeable subset of applicants are
invited to interview (paying attention to demographics already at this
stage); then a meeting is held to decide on who to hire, again with
diversity metrics explicitly in mind.

\item {\bf Measurement:} Measuring success in this effort is
relatively easy.  In Fall 2022, for example, Pierce taught the class
with 51 TAs, including 26 women and 5 URMs (Black or Latinx).  By
comparison, Penn's undergraduate population is 53\% female and 18\%
self-identified Black or Latinx; i.e., we did well on gender balance
and we still have work to do on ethnic diversity.  We will use the
same metrics during the period of this grant.

\item {\bf PI Engagement:} PI Pierce (along with Stephanie Weirich and
Steve Zdancewic) designed the 1200 course content, and he continues to
teach it regularly.
\end{enumerate}


% What are the components of a meaningful BPC plan?

% For both Connected and Standalone Project BPC Plans, a meaningful BPC plan should answer positively to the following five elements:

% Goal and Context: Does the plan describe a goal and the data from your institution(s) or local community that justify that goal?
% Intended population(s): Does the plan identify the characteristics of participants from an underrepresented group listed above, including school level (e.g., African-American undergraduates or female high-school students)?
% Strategy: Does the plan describe activities that address the goal(s) and intended population(s)?
% Measurement: Is there a plan to measure the outcome(s) of the activities?
% PI engagement: Is there a clear role for each PI and co-PI? Does the plan describe how the PI is prepared (or will prepare or collaborate) to do the proposed work?

\subsection*{2. Draft Departmental BPC Plan}

\subsection*{Context}

The University of Pennsylvania is a private, Ivy League research university located in Philadelphia, Pennsylvania. Undergraduate admission is highly selective, with 9\% of applicants admitted in 2020. The table below presents some data on the demographics of the undergraduate population of the CIS Department in 2016 and in 2021, comparing the percentage of majors who identify as women and historically under-represented minorities (URM) in computer science, namely as Hispanic, American Indian or Alaska Native, Black or African American, or Native Hawaiian or Pacific Islander. Data for the entire undergraduate population of the university, and from the Taulbee survey of US and Canadian computer science departments, are also presented as points of comparison.

\begin{center}
\begin{tabular}{|l|l|l|l|l|}
\hline
  Population & \multicolumn{2}{|c|}{Female} & \multicolumn{2}{|c|}{URM}
\\
\hline
  & 2016 & 2021 & 2016 & 2021
\\
\hline
Penn overall &
50.2\% &
53.6\% &
17.7\% &
18.2\%
\\
Penn CIS &
14.8\% &
27.9\% &
5.9\% &
9.4\%
\\
Taulbee &
17.9\% &
22.3\% &
11.1\% &
12.7\% \\
\hline
\end{tabular}
\end{center}

The CIS Department is above the Taulbee averages of other CS departments in terms of gender diversity, though it lags far behind Penn as a whole. In URM representation, the department unfortunately trails both the Taulbee averages and the rest of the university, though these gaps have narrowed since 2016 even as the department has undergone substantial growth in enrollment from 169 graduating seniors in 2016 to 297 students in 2021.

\subsection*{Goals}

We have identifed four key goals for the department to continue to
improve diversity.
\begin{enumerate}[itemsep=-1mm]
\item
Increase diversity of undergraduate students, demonstrating year on year improvements toward matching and then exceeding national averages.
\item
Increase diversity of graduate students, demonstrating year on year improvements toward matching and then exceeding national averages.
\item
Increase diversity of faculty, demonstrating year on year improvements toward matching and then exceeding national averages.
\item
Enhance local area CS pipeline with K-12 outreach.
\end{enumerate}

\subsection*{Activities and Measurement}

% \paragraph*{Goal 1a: Increase diversity of undergraduate students engaged in PL
% / HCI research.}

% \textbf{Goal and Context}:

% \textbf{Intended population(s)}:

% \textbf{Strategy}:

% \textbf{Measurement}:

% \textbf{PI engagement}: (Maybe this will be implied by ``Strategy''.

% \paragraph*{Goal 1b: Increase diversity of undergraduate students poised to
% bring verification techniques into their professional software work.}

% \textbf{Goal and Context}:

% \textbf{Intended population(s)}:

% \textbf{Strategy}:

% \textbf{Measurement}:

% \textbf{PI engagement}: (Maybe this will be implied by ``Strategy''.

% \paragraph*{Goal 1c: Increase representation of undergraduates from
% underrepresented minorities in teaching positions in PL / HCI.}

% \textbf{Goal and Context}:

% \textbf{Intended population(s)}:

% \textbf{Strategy}:

% \textbf{Measurement}:

% \textbf{PI engagement}: (Maybe this will be implied by ``Strategy''.

% \paragraph*{Goal 2a: Increase diversity of graduate students engaged in PL / HCI
% research in the PIs' research groups.}

% \textbf{Goal and Context}:

% \textbf{Intended population(s)}:

% \textbf{Strategy}:

% \textbf{Measurement}:

% \textbf{PI engagement}: (Maybe this will be implied by ``Strategy''.

% \paragraph*{Goal 2b: Increase diversity of graduate students engaged in PL / HCI
% research in the department broadly.}

% \textbf{Goal and Context}:

% \textbf{Intended population(s)}:

% \textbf{Strategy}:

% \textbf{Measurement}:

% \textbf{PI engagement}: (Maybe this will be implied by ``Strategy''.

% \paragraph*{Goal 4: Increase engagement of K--12 students from underrepresented
% groups in computing education.}

% I have less interesting to say here than in the other categories than to mention
% our current outreach efforts, and the goals of those outreach programs. (I
% embarrassingly am not currently involved in outreach efforts at Penn).

% ---

\paragraph*{Goal 1: Undergraduate Student Diversity}

\begin{enumerate}[itemsep=-1mm]
\item Activity: Support participation in conferences such as Grace Hopper Celebration of Women in Computing, Richard Tapia Celebration of Diversity in Computing, Capitol WIC, (Bhusnurmath, Sheth)
\item Activity: Hold regular department-wide Diversity Summits to discuss representation and solicit feedback. Recent Diversity Summit feedback has led to greater transparency in TA hiring.
\item Activity: Engage undergraduate researchers through Center for Undergraduate Research \& Fellowships, summer visiting scholar positions and NSF REUs.
\item Measurement: Evaluate representation in the major with particular emphasis on entry and retention statistics in the introductory CS courses.
\item Measurement: Evaluate number of undergraduate researchers hosted with particular emphasis on their success in the major as well as their applications and matriculations to graduate school. Evaluate the number of faculty hosting undergraduate researchers.
\end{enumerate}

\paragraph*{Goal 2: Graduate Student Diversity}
\begin{enumerate}[itemsep=-1mm]
\item Activity: Better promote fellowships to faculty. Leverage internal opportunities like Fontaine Fellowships for recruiting PhD students from underrepresented groups
\item Activity: Allow more flexibility on PhD milestone exams to support students from diverse backgrounds with heterogeneous preparation.
\item Activity: Graduate student mentoring workshops like Programming Languages Mentoring Workshop, Women in Theory, Young Architect Workshop (Angel, Devietti, Pierce, Rabin, Weirich, Zdancewic)
\item Measurement: Evaluate the number of graduate students participating in workshops and conferences that celebrate diversity.
\end{enumerate}

\paragraph*{Goal 3: Faculty Diversity}

\begin{enumerate}[itemsep=-1mm]
\item Activity: Engage with EECS Rising Stars and NSF CI Fellows by (a) encouraging Penn’s graduate students to apply, (b) encouraging faculty candidates from these programs to apply to Penn's searches.
\item Activity: Diversify symposium and seminar speakers.
\item Measurement: Evaluate demographics of faculty applications, short-lists, interviews, and offers. Establish baselines for assessing year-over-year improvement.
\item Measurement: Evaluate demographics of seminar speakers.
\end{enumerate}

\paragraph*{Goal 4: Improve K-12 Outreach}

\begin{enumerate}[itemsep=-1mm]
\item Activity: Fife Academy provides CS education in K-8 schools
\item Activity: Steppingstone Scholars Blended Learning Initiative offers summer programming camp for 6th-12th grade students
\item Activity: After-school and summer courses preparing for AP CS A exam (Fouh)
\item Activity: Tech It Out Philly high school coding workshops (Haeberlen)
\item Activity: WICS High School Day for Girls
\item Measurement: monitor number of participating Philadelphia and Penn students
\item Measurement: track AP CS A exam results
\end{enumerate}
