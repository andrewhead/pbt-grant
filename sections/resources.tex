\section*{Facilities, Equipment, and Other Resources}

\subsection*{Facilities and Equipment}

Penn's departmental computing facilities include a standard collection of
networking services, printers, fileservers, and compute servers.

We will purchase a compute server, to be set up in the department’s
machine room, for running larger empirical studies (e.g., the
benchmarking study in Y1-Y2 and the generator automation task in
Y4-Y5), as described in the Budget Justification.

No other specialized facilities or equipment will be needed to carry
out the proposed work.


\subsection*{Unfunded Collaborations}

We have initiated unfunded collaborations with four external experts,
who will help with various project tasks.
\begin{itemize}
\item Professor John Hughes (Chalmers University, Sweden), a longtime
collaborator of PI Pierce and one of the
QuickCheck creators, will collaborate on the projects involving
generation, as described in
\sectionref{sec:gen}.
\item Professor Hila Peleg (Technion, Israel), an expert with
experience with both PBT and program
synthesis, will collaborate on generator automation
\sectionref{sec:reflective}.
\item Professor Leonidas Lampropoulos (University of Maryland), an
expert on PBT tools, will collaborate on generator benchmarking
\sectionref{sec:benchmarking}.
\item Zac Hatfield-Dodds, the main maintainer of the Python Hypothesis
tool, will help us interface with the Python community and incorporate
project results into Hypothesis \sectionref{sec:nurturing}.
\end{itemize}
Letters of collaboration from all four can be found in the supplemental
documents.
