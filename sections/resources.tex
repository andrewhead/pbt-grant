\section*{Facilities, Equipment, and Other Resources}

\subsection*{Facilities and Equipment}

Penn's departmental computing facilities include a standard collection
of networking services, printers, fileservers, and compute
servers. The proposed budget includes up-to-date laptops for project
personnel.

% Aside from
% these,
% no specialized facilities or equipment will be needed to carry
% out the proposed work.

\smallskip

This project will make use of Penn HCI's usability study lab to host lab-based usability evaluations of programming languages and interfaces. PI Head is the co-founder and co-lead of Penn HCI, as well as its accompanying lab space. The Penn HCI Lab is a 500-square-foot space devoted to the conduct of usability studies. The keypad-protected lab is located on the second floor of Levine Hall, Penn's computer science building. The lab is conveniently located a two-minute walk from Penn HCI's graduate student and faculty offices. The lab also resides on Penn's main campus, bringing it within walking distance of the Penn student body, staff and faculty, as well as the broader West Philadelphia community. The space was designed to support high throughput by enabling the conduct of multiple study sessions in parallel in each of its three study rooms. It was also designed to support flexibility in study design with reconfigurable study spaces. The space provides:

\textbf{Three usability study rooms}. Each of the three rooms measures approximately 100 square feet. Each is equipped with a 27'' wall-mounted monitor, a 4' long desk, and 3 rolling chairs for participants and/or observers. All furniture is movable, permitting configuration of the space to support both desktop-based studies, as well as studies involving interactions in physical space. A camera is mounted to the ceiling in each room, providing live video and audio feed and recording capabilities. The camera can pan and zoom in response to input from a control room. Each room is sound-isolated; windows in the door are frosted to preserve participant anonymity.

\textbf{Control room}. Experimenters can monitor usability study sessions from a control room adjacent to the study rooms and hidden from participants' view. This is particularly useful should an experimenter wish to conduct study sessions in multiple rooms in parallel, or if an experimenter wishes to give participants space while they complete their tasks. It receives live audio and video feeds from each of the 3 study rooms. Participants can be observed on a 6' wall-mounted display. The control room also has an 8'-long table top where experimenters can organize study materials (e.g., questionnaires, consent forms) in a centralized location out of sight of the individual rooms.

\textbf{Recording equipment}. Experimenters can record video and audio from all usability study rooms. Video and audio feeds from all study rooms are streamed to a laptop in the lab's electronics closet. Recordings can be made by inserting a USB stick into this laptop in advance of a study session and launching the lab's recording software. Recordings can be made for multiple rooms at the same time.

\textbf{Study equipment}. To help experimenters conduct studies, our lab also has one Mac Mini, one Tobii Pro Spark eye tracker, one USB keyboard, and one USB mouse. Experimenters may supplement this equipment with their own laptop and devices, which they may use as the study device by plugging into the monitor in any study room. This equipment has sufficed for studies to date. Our inventory of equipment will continue to be expanded to support the needs of upcoming experiments.

\subsection*{Unfunded Collaborations}


\noindent {\bf John Hughes} is the co-creator of the QuickCheck framework in
Haskell, which popularized PBT. He is the CEO of QuviQ, a startup that
implements PBT in industry, and a professor of computer science at Chalmers
University.  Prof. Hughes is a longtime collaborator of PI
Pierce, and has specific expertise in the kinds of generator-focused projects
appearing in \sectionref{sec:gen} of the Project Description.

\smallskip\noindent {\bf Hila Peleg} is an assistant professor at the Technion with expertise in both
PL and HCI. She is an external thesis committee member for one of PI Pierce's current
students, and has expressed interest in many of the proposal
projects. Prof. Peleg's research
focuses on program synthesis, so she will be an invaluable ally for the work in
\sectionref{sec:reflectivepeople}.

\smallskip\noindent {\bf Leonidas Lampropoulos} is an assistant professor at the
University of Maryland and a former student of PI Pierce.  He is an expert on
PBT in the context of proof assistants like Coq, and has collaborated
successfully with PI Pierce and his students for many years. Prof. Lampropoulos
is currently involved in the beginnings of the project in
\sectionref{sec:benchmarks}.

\smallskip\noindent {\bf Zac Hatfield-Dodds} is the main maintainer of
Hypothesis, the most popular PBT library in Python (and likely the
most popular PBT library full stop). He has expressed interest in many
of the expected outcomes of the tasks in the Project Description,
including especially \sectionref{sec:reflectivepeople},
\sectionref{sec:benchmarks}, \sectionref{sec:ghostwriter},
\sectionref{sec:evaluating_distributions}, and \sectionref{sec:ide}.

\smallskip\noindent {\bf Chris Callison-Burch} is an associate professor of
Computer and Information Science at the University of Pennsylvania.
%
He is best known for his research into natural language
processing. His current research is focused on applications of large
language models to long-standing challenges in artificial
intelligence. In the fall of 2023, at the suggestion of PI Pierce, he
is leading a seminar course on the intersection of LLMs, programming
languages, and software engineering. Prof. Callison-Burch will advise
on the development of programming assistants that incorporate LLMs
technologies as part of an integrated set of technologies for the
ideation and implementation of specifications as described in \sectionref{sec:ghostwriter}.

\bigskip

\noindent Letters of collaboration from all five can be found in the supplemental
documents.
