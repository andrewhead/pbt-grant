\section*{Facilities, Equipment, and Other Resources}

Penn's departmental computing facilities include a standard collection of
networking services, printers, fileservers, and compute servers.  No
specialized facilities or equipment will be needed to carry out the
proposed work.

\bcp{Talk about the server machine (use text from the budget justification)}

\discuss{
Document all three (four?) unfunded external collaborations.
  \begin{itemize}
  \item Any substantial collaboration with individuals not included in
  the budget should be described in the Facilities, Equipment and
  Other Resources section of the proposal (see Chapter II.D.2.g) and
  documented in a letter of collaboration from each collaborator. Such
  letters should be provided in the supplementary documentation
  section of Research.gov and follow the format instructions specified
  in Chapter II.D.2.i.
  \item We also need to make sure that the appropriate sections of the
  project description mention the collaborations.
  \end{itemize}
}

\noindent {\bf John Hughes} is the co-creator of the QuickCheck framework in
Haskell, which popularized PBT. He is the CEO of QuviQ, a startup that
implements PBT in industry, and a professor of computer science at Chalmers
University.  Prof. Hughes is a longtime collaborator of PI
Pierce, and has specific expertise in the kinds of generator-focused projects
appearing in \sectionref{sec:gen} of the Project Description.

\smallskip\noindent {\bf Hila Peleg} is an assistant professor at the Technion with expertise in both
PL and HCI. She is an external thesis committee member for one of PI Pierce's current
students, and has expressed interest many of the proposal projects. Prof. Peleg
focuses on program synthesis, so she will be an invaluable ally for the work in
\sectionref{sec:genauto}.

\smallskip\noindent {\bf Leonidas Lampropoulos} is an assistant professor at the
University of Maryland and a former student of PI Pierce.  He is an expert on
PBT in the context of proof assistants like Coq, and has collaborated
successfully with PI Pierce and his students for many years. Prof. Lampropoulos
is currently involved in the beginnings of the project in
\sectionref{sec:benchmarks}.

\smallskip\noindent {\bf Zac Hatfield-Dodds} is the main maintainer of
Hypothesis, the most popular PBT library in Python (and possibly the most
popular one overall). He has expressed interest in many of the expected outcomes
of the tasks in the Project Description, including especially
\sectionref{sec:nurturing}.