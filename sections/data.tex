\section*{Data Management Plan}

The project will produce four types of data.

\begin{enumerate}
\item Software artifacts,
including implementations, test cases, software revision histories, and
other items.  These will be distributed under an open-source
license.  They will be documented so as to allow others to understand,
use, and modify them.  (Both PIs have extensive records of
distributing artifacts in this way.)  Artifacts will be made available
to the public on Github and maintained for at least three years beyond
the end of the project lifecycle.

\item Technical papers and talks describing our experiences
and results.  These will be published in academic conferences and journals.
Preprints will be made available on Arxiv.

\item Educational materials aimed at undergraduates, masters students,
and professional software developers. These will be made available on the
project's public website under an open-source license.

\item Raw data gathered during user studies, which will be kept
confidential and stored securely in accordance with IRB-approved best
practices. We elaborate about this below.
\end{enumerate}

% \iflater{%
% \amh{Consider integrating the following, which Andrew wrote up for his NSF
% CAREER to address more of the concerns the data management plan is asked to
% address.}
% \amh{This is fine to include. If we do, update course numbers to CIS 120.}

\paragraph{Data from user studies}
As part of tool design and
evaluation activities, we will collect human subjects data in the form of questionnaire responses, interface usage
logs, and, during lab evaluations, screen recordings, audio recordings, and
researcher notes. Data will be collected from users in the following settings: (1)
in-lab usability studies; (2) interviews and observations with developers;
(3) surveys; (4) student learners in CIS 1200 and CIS 5730; and (5) usage of the tools
in the wild.

All data from human subjects will be collected, stored, and analyzed in accordance
with institutional best practices following procedures that have undergone review by Penn's Institutional
Review Board. In all circumstances, users' express consent will be obtained
prior to collection and analysis of data. Data will be handled in a way that is sensitive
to the following considerations:

\emph{Confidentiality}. User confidentiality will be preserved by collecting
only those identifiers absolutely necessary to manage studies and conduct the required
analyses. Identifiers that need to be collected will be removed from the data when it
becomes possible to do so: for instance, audio recordings will be transcribed and then
deleted to remove the identifier of participants' voices, and textual identifiers (like
names and company affiliations) will be removed from transcripts. Participants will
be able to request that we delete or redact usability data should it contain information that is sensitive for them. Any transcription services that we use will have signed strict confidentiality agreements.

\emph{Security}. Human subjects data will be collected, stored, and analyzed on password-protected,
encrypted machines managed by research personnel who have undergone human subjects research
training. The data may also be stored on secured departmental servers and password-protected
shared folders hosted in university-approved cloud services.
We will retain data long enough that the Ph.D. student researchers can consult it throughout their entire programs of study. Our plan is to retain study data for seven years following the project start date. After that time, the data will be deleted.
}
