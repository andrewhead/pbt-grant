\iflater
\section*{Data Management Plan}

\todo{GPG: In the interest of completeness and transparency, PIs must
  describe, as part of their Data Management Plans, how they will
  provide access to well-documented datasets, modeling and/or
  simulation tools, and code bases to support
  reproducibility/replicability of their methods and results for a
  reasonable time beyond the end of the project lifecycle.  (Check
  this!)
  \\
  For additional information on the Dissemination and Sharing of Research Results, see: \url{https://www.nsf.gov/bfa/dias/policy/dmp.jsp}.
\\
For specific guidance for Data Management Plans submitted to the Directorate for Computer and Information Science and Engineering (CISE) see: \url{https://www.nsf.gov/cise/cise_dmp.jsp}.
}

\todo{GPG: Proposals must include a document of no more than two pages uploaded under “Data Management Plan” in the supplementary documentation section of Research.gov. This supplementary document should describe how the proposal will conform to NSF policy on the dissemination and sharing of research results (see Chapter XI.D.4), and may include:
  \begin{itemize}
  \item
the types of data, samples, physical collections, software, curriculum materials, and other materials to be produced in the course of the project;
\item the standards to be used for data and metadata format and content (where existing standards are absent or deemed inadequate, this should be documented along with any proposed solutions or remedies);
\item policies for access and sharing including provisions for appropriate protection of privacy, confidentiality, security, intellectual property, or other rights or requirements;
\item policies and provisions for re-use, re-distribution, and the production of derivatives; and
\item plans for archiving data, samples, and other research products, and for preservation of access to them.
  \end{itemize}
}

The proposed project will produce two types of data: (1) software artifacts,
including implementations, test cases, software revision histories, and
other items; and (2) technical papers and talks describing our experiences
and results.

All software artifacts will be distributed under an open-source
license.  They will be documented so as to allow others to understand,
use, and modify them.  (Both PIs have extensive records of
distributing artifacts in this way.)  Artifacts will be made available
to the public on Github and maintained for at least three years beyond
the end of the project lifecycle.

Technical papers will be published in academic conferences and journals.
Drafts will be made available on Arxiv.

\bcp{Say something about educational materials.}

\fi